\documentclass{beamer}
\setbeamertemplate{caption}{\insertcaption}
\usepackage{cancel}
\renewcommand\CancelColor{<color command>}
\usepackage{xcolor}
\newcommand\Ccancel[2][black]{\renewcommand\CancelColor{\color{#1}}\cancel{#2}}

\definecolor{bluenewcolor}{RGB}{20,20,100}
\definecolor{greennewcolor}{RGB}{0,100,0}
\definecolor{decisionColor}{HTML}{DEE1B2}
\definecolor{nodeColor}{HTML}{EBFAFF}
\def\mathunderline#1#2{\color{#1}\underline{{\color{black}#2}}\color{black}}
\usetheme{metropolis}           % Use metropolis theme
\title{Master thesis presentation\\
STOCHASTIC APPROACH TO MANY-BODY PROBLEMS}
\date{\today}
\author{Anna Gribkovakaya}
\institute{University of Oslo}
\begin{document}
  \maketitle
  \section{Introduction}
   \begin{frame}{Outline}
   \begin{itemize}
   	\item Many body \textit{Ab initio} methods: deterministic and stochastic
   	\item Many-body problem formulation
   	\item Coupled Cluster Theory
   	\item Coupled Cluster Quantum Monte Carlo
   	\item Results
   	\item Summary
   \end{itemize}
 \end{frame}
   \begin{frame}{Objectives}
	The main objectives:
	\begin{itemize}
 	\item Implement numerical methods to solve the Time Independent Schr\"{o}dinger equation for N electrons
 	\item Study the Homogeneous Electron gas using Coupled Custer approach
 	\item Study the Homogeneous Electron gas using the Coupled Custer Monte Carlo approach
 	\item Compare the two methods
 	\item Develop a code that can be used for other systems (Quantum Dots)
	\end{itemize}
	\end{frame}


  \section{Main Part}
  
  \begin{frame}{Many-body: \textit{Ab initio} quantum many body methods}
  \uncover<1->{Deterministic or Wave Function Methods}
  \uncover<1->{ \begin{itemize}
  		\item Hatree-Fock Method
  		\item Many Body Perturbation Theory
  		\item Full Configuration Interaction 
  		\item Coupled Cluster Method
  \end{itemize}}
   \uncover<2->{Stochastic or Monte Carlo (MC) Methods}
  	  \uncover<2->{	\begin{itemize}
  		\item Variational MC
  		\item Diffusion MC
  		\item Full Configuration MC
  		\item Coupled Cluster MC
  	\end{itemize}}
  \uncover<3->{Density Functional Theory - widely used, but has issues}
   \end{frame}
  \begin{frame}{Many-body: problem formulation}
The Hamiltonian for quantum mechanical system consisting of N particles  can be written as
\begin{equation*}
\hat{H} = -\frac{1}{2}\sum_{i=1}^{N} \nabla_i^2 + \sum_{i<j}^{N}\frac{1}{r_{ij}} - \sum_{i=1}^{N} \hat{V}(r_i),
\end{equation*}
$r_{ij}$  relative distance between the particles, \\
$\hat{V}(r_i)$  potential that represents the system under consideration.\\

Stationary Schr\"{o}dinger equation can be written as:
\begin{equation*}
\hat{H}\Psi_n(\vec{\textbf{R}}) = E_n\Psi_n(\vec{\textbf{R}}),
\end{equation*}
$\vec{\textbf{R}}$ - vector representing both coordinates and spins for all particles.
\begin{equation*}
\vec{\textbf{R}} = \{\vec{R_n}\}= \{(\vec{r}, \sigma)_n\},
\end{equation*}
For fermions the wave function must be antisymmetric:
\begin{equation*}
\Psi_n({\dots, \vec{R}_p , \dots, \vec{R}_q, \dots}) = -\Psi_n({\dots, \vec{R}_q , \dots, \vec{R}_p, \ \dots}),
\end{equation*}
\end{frame}
  \begin{frame}{Many-body: Slater Determinant Space}
  \begin{itemize}
  	\item Slater Determinant (SD) Space is a Hilbert space for fermions 
  	\item Each SD is constructed from $N$ orthonormal single electron
  	wavefunctions
  	
  	\item SD for reference vacuum state :
 
  \begin{equation*}
  |D_0\rangle = \frac{1}{\sqrt{N!}}
  \left| \begin{array}{ccccc} \psi_{1}(x_1)& \psi_{1}(x_2)& \dots & \psi_{1}(x_N)\\
  \psi_{2}(x_1)&\psi_{2}(x_2)& \dots  & \psi_{2}(x_N)\\  
  \vdots & \vdots & \ddots  & \vdots \\
  \psi_{N}(x_1)&\psi_{N}(x_2)& \dots  & \psi_{N}(x_N)\end{array} \right| 
  \end{equation*}
  \item Every  SD is anti-symmetric by construction
   \end{itemize}
  \end{frame}

\section{Coupled Cluster Theory}

  \begin{frame}{Coupled cluster theory: Exponential ansatz}
	The approximation for wave function for Coupled Cluster is:
	\begin{equation*}
	\Psi_{CC} =  e^{\hat{T}}|D_0\rangle,
	\end{equation*}
	\begin{equation*}
	\hat{T} = \sum_i^{N} \hat{T}_i  = \hat{T}_1 + \hat{T}_2 +\hat{T}_3 + \dots + \hat{T}_N,
	\end{equation*}
	Each excitation operator can be written in terms of creation and annihilation operators:
	\[ 
	\hat{T}_1 = \sum_{i}\hat{t}_i = \sum_{ia}t_{i}^{a} c^\dag_{a}  c_{i}, \text{    } 
	\hat{T}_2 = \sum_{i<j}\hat{t}_{ij}^{ab} = \frac{1}{2!^2}\sum_{ijab}t_{ij}^{ab} c^\dag_{a} c^\dag_{b} c_{j} c_{i}, 
	\]
	\[ 
	\hat{T}_3 = \sum_{i<j<k}\hat{t}_{ijk}^{abc} = \frac{1}{3!^2}\sum_{ijkabc}t_{ijk}^{abc} c^\dag_{a} c^\dag_{b} c^\dag_{c} c_{k}c_{j} c_{i}.
	\,
	\]
	here $i,j,k$ indexes denote the orbitals that are occupied in the reference determinant and $a,b,c$ denote those that are not.
	\end{frame}

  \begin{frame}{Many-body: coupled cluster method}
  An orthonormal set of 2M spin-orbitals, where N are occupied.
\begin{equation*}
\binom {2M}{N} = \frac{N!}{2M!(2M-N)!}, 
\end{equation*}

For the electron gas:\\

$N$=14, $2M$=3/38	$\rightarrow$ $10^{10}$\\
$N$=14, $2M$=10/246	$\rightarrow$ $10^{22}$\\
$N$=14, $2M$=20/730	$\rightarrow$ $10^{29}$

\end{frame}
\begin{frame}
\frametitle{Coupled Cluster: What about the amplitudes?}
Insert the coupled cluster wave function in the TISE:
\begin{equation*}
\hat{H}e^{\hat{T}}|D_0\rangle = Ee^{\hat{T}}|D_0\rangle.	
\end{equation*}
Linked coupled cluster equations:
\begin{equation*}
\langle D_0|e^{-\hat{T}}\hat{H}e^{\hat{T}}|D\rangle = E,
\end{equation*}
\begin{equation*}
\langle D_{ij\dots}^{ab\dots}|e^{-\hat{T}}\hat{H}e^{\hat{T}}|D_0 \rangle = 0.
\end{equation*}

\end{frame}


\section{Coupled Cluster Quantum Monte Carlo}
\begin{frame}
\frametitle{Projector Monte Carlo}
Start with TDSE:
\begin{equation*}
-i\hbar \frac{\partial}{\partial t} |\Psi(t)\rangle= \hat{H} |\Psi(t)\rangle,
\end{equation*}
Wick rotation of time to the standard TDSE:
\begin{equation*}
-\frac{\partial}{\partial \tau} |\Psi(\tau)\rangle= \hat{H} |\Psi(\tau)\rangle,
\end{equation*}
Given a starting wavefunction $|\Psi^{(\tau = 0)}\rangle$, the wavefunction for arbitrary $\tau$ can be expressed with a propagator:
\begin{equation*}
|\Psi^{(\tau)}\rangle \propto e^{-\tau \hat{H}}|\Psi^{(\tau = 0)}\rangle,
\end{equation*}
\end{frame}

\begin{frame}
\frametitle{Projector Monte Carlo}
Assuming that initial wave function has nonzero overlap with the ground state of the Hamiltonian:
\begin{equation*}
|\Psi_0\rangle = \lim_{\tau \rightarrow \infty} e^{-\tau (\hat{H}-S)}|\Psi^{(\tau = 0)}\rangle,
\end{equation*}
Approximating the exponential propagator by repeated application of the linear propagator:
\begin{equation*}
|\Psi_0\rangle = \lim_{N \rightarrow \infty} \Big[1 - \delta \tau (\hat{H}-S)\Big]^N |\Psi^{(\tau = 0)}\rangle,
\end{equation*}
The ground state wavefunction is obtained by an iterative application of a projection operator for a sufficient number of times:
\begin{equation*}
|\Psi^{(\tau + \delta \tau)}\rangle = (1 - \delta \tau(\hat{H}-S))| \Psi^{(\tau)}\rangle.
\end{equation*}
\end{frame}

\begin{frame}
\frametitle{Coupled Cluster Quantum Monte Carlo}
\begin{equation*}
|\Psi^{(\tau + \delta \tau)}\rangle = (1 - \delta \tau(\hat{H}-S))| \Psi^{(\tau)}\rangle,
\end{equation*}
\uncover<1->{Use the exponential ansatz for wavefunction and project equation on the excited determinant $D_{\{\boldsymbol{i}\}}$:}
\uncover<1->{\begin{align*}
	\langle D_{\{\boldsymbol{i}\}}|\Psi_{CC}^{(\tau + \delta \tau)}\rangle = \langle D_{\{\boldsymbol{i}\}}|1 - \delta \tau (\hat{H}-S)|\Psi_{CC}^{(\tau)}\rangle,\\
	\langle D_{\{\boldsymbol{i}\}}|e^{\hat{T}}D_0\rangle = \langle D_{\{\boldsymbol{i}\}}|e^{\hat{T}}D_0\rangle - \delta \tau \langle D_{\{\boldsymbol{i}\}}|(\hat{H}-S)e^{\hat{T}}|D_0\rangle,
	\end{align*}}
\uncover<2->{Projection of $\Psi_{CC}$ onto a single determinant contains the amplitude of the excitor for that determinant and terms involving multiple amplitudes.}
\uncover<2->{\begin{equation*}
	\langle D_{\boldsymbol{i}}|e^{\hat{T}^{(\tau)}}D_0\rangle = t_{\boldsymbol{i}}^{(\tau)} + \mathcal{O}(\hat{T}^2),
	\end{equation*}}
\uncover<3->{\begin{equation*}
	t_{\{\boldsymbol{i}\}}^{(\tau+\delta\tau)}  \only<3>{+\mathcal{O}(\hat{T}^2)\phantom{2}}\only<4>{+\Ccancel[red]{\mathcal{O}(\hat{T}^2)}\phantom{2}}\only<5>{} = t_{\{\boldsymbol{i}\}}^{(\tau)}  \only<3>{+\mathcal{O}(\hat{T}^2)\phantom{2}}\only<4>{+\Ccancel[red]{\mathcal{O}(\hat{T}^2)}\phantom{2}}\only<5>{} - \delta \tau \langle D_{\{\boldsymbol{i}\}}|(\hat{H}-S)e^{\hat{T}}|D_0\rangle,
	\end{equation*}}
%\uncover<7->{\begin{equation*}
%\frac{\delta t_{\{\boldsymbol{i}\}}}{\delta\tau} = -\langle D_{\{\boldsymbol{i}\}}|(\hat{H}-S)e^{\hat{T}}|D_0\rangle.
%\end{equation*}}
\end{frame}

	
  \begin{frame}{Many-body Systems: QD and HEG}
	equations
	\end{frame}
  \begin{frame}{Results: Deterministic CC}
	tables
	\end{frame}
\section{Results}
  \begin{frame}{Results: MQ CC}
	tables
	\end{frame}
\section{Conclusion}
  \begin{frame}{Summary}
	bla
	\end{frame}
\end{document}